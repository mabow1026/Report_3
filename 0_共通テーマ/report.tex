\documentclass[11pt,dvipdfmx]{jarticle}

\usepackage{eee}

\begin{document}
% トップページを書く

\begin{jikkenTitle}
  \gakunen{3} % 学年を記述。この行で全体の枠を表示
  \numTitle{0}{共通テーマ(1)} % 実験番号、タイトルを記述
  \subTitle{} % サブタイトルがあれば記述
  \jikkenbi{令和4年4月14日(木)} % 実験日を記述
  %  \jikkenbiII{令和4年y月zz日(aa)} % 実験日を記述(二日目がある場合。ない場合はこの行をコメントアウト)
  \kyoudou{共同実験者名} % 共同実験者名を記述
  \yoteibi{MM/DD}% 予定日を記述
  \hanNumberName{3}{3333}{宮崎 永} % 班番号・学生番号・氏名を記述。この行でタイトルページの描画を終了
\end{jikkenTitle}

\section{目的}
\section{原理}
\section{方法}
\subsection{実験回路}
\label{sec:zikkenkairo}
VSCodeの拡張機能のdraw.ioを用いてRLC直列回路の実験回路図を作成した。これを図\ref{fig:RLC直列}に示す。
また、RLC並列回路の場合の実験回路図を図\ref{fig:RLC並列}に示す。
\begin{figure}[htbp]
  \begin{minipage}[b]{0.45\linewidth}
    \centering
    \includegraphics[keepaspectratio, scale=0.5]{RLC.drawio.pdf}
    \caption{RLC直列回路}
    \label{fig:RLC直列}
  \end{minipage}
  \begin{minipage}[b]{0.45\linewidth}
    \centering
    \includegraphics[keepaspectratio, scale=0.5]{RLC2.drawio.pdf}
    \caption{RLC並列回路}
    \label{fig:RLC並列}
  \end{minipage}
\end{figure}



\subsection{実験手順}
\label{sec:zikkentejun}
図\ref{fig:RLC直列}及び図\ref{fig:RLC並列}の回路を用いて、電圧源の周波数を変化させ、素子の端子電圧と電流を測定する実験を行った。
以下にその手順を示す。なお、RLC直列回路の抵抗は10$\Omega$、インダクタは100mH、キャパシタ0.1$\mu$Fの物を使用した。
RLC並列回路の抵抗は10$\Omega$、インダクタは10$\mu$H、キャパシタ47$\mu$Fの物を使用した。
\subsubsection{RLC直列回路の実験手順}
\label{sec:RLC直列}
\begin{enumerate}[1)]
  \item 電源から正弦波交流を出力し、電圧のピーク値を20Vに設定する。
  \item 電源の周波数を500Hzから3.5kHzまで200Hz刻みで変化させ、それぞれ各箇所の電圧計と電流計を読み取って記録する。
\end{enumerate}
\subsubsection{RLC並列回路の実験手順}
\label{sec:RLC並列}
\begin{enumerate}[1)]
  \item 電源から正弦波交流を出力し、電圧のピーク値を20Vに設定する。
  \item 電源の周波数を1kHzから15kHzまで1kHz刻みで変化させ、それぞれ各箇所の電圧計と電流計を読み取って記録する。
\end{enumerate}


\section{結果}
\subsection{RLC直列回路の実験結果}
\ref{sec:RLC直列}で行った実験の結果を整理し、インピーダンス及び偏角を算出した。算出に用いた式を以下に示す。
\begin{equation}
  |Z| = \sqrt{R^2 + (wL - \frac{1}{wC})^2}
\end{equation}
\begin{equation}
  \phi = \arctan \left(\frac{wL - \frac{1}{wC}}{R}\right)
\end{equation}
また、算出した値をグラフにしたものをそれぞれ図\ref{fig:RLC直列_インピーダンス}と図\ref{fig:RLC直列_偏角}に示す。
さらに、インピーダンスの軌跡を複素平面に描画したものを図\ref{fig:RLC_軌跡1}に示す。
\begin{figure}[htbp]
  \centering
  \includegraphics[keepaspectratio, scale=0.5]{RLC_Series_Z.png}
  \caption{RLC直列回路のインピーダンス}
  \label{fig:RLC直列_インピーダンス}
\end{figure}
\begin{figure}[htbp]
  \begin{minipage}[b]{0.45\linewidth}
    \centering
    \includegraphics[keepaspectratio, scale=0.5]{RLC_Series_Arg.png}
    \caption{RLC直列回路の偏角}
    \label{fig:RLC直列_偏角}
  \end{minipage}
  \begin{minipage}[b]{0.45\linewidth}
    \centering
    \includegraphics[keepaspectratio, scale=0.5]{RLC_Series_Locus.png}
    \caption{RLC直列回路のインピーダンスの軌跡}
    \label{fig:RLC_軌跡1}
  \end{minipage}
\end{figure}
\newpage

\subsection{RLC並列回路の実験結果}
\ref{sec:RLC並列}で行った実験の結果を整理し、インピーダンス及び偏角を算出した。算出に用いた式を以下に示す。
\begin{equation}
  |Z| = \frac{1}{\sqrt{\frac{1}{R^2} + (\frac{1}{wL} - wC)^2}}
\end{equation}
\begin{equation}
  \phi = \arctan \left(R \left(\frac{1}{wL} - wC \right) \right)
\end{equation}
また、算出した値をグラフにしたものをそれぞれ図\ref{fig:RLC並列_インピーダンス}と図\ref{fig:RLC並列_偏角}に示す。
さらに、インピーダンスの軌跡を複素平面に描画したものを図\ref{fig:RLC_軌跡2}に示す。
\begin{figure}[htbp]
  \centering
  \includegraphics[keepaspectratio, scale=0.5]{RLC_Parallel_Z.png}
  \caption{RLC並列回路のインピーダンス}
  \label{fig:RLC並列_インピーダンス}
\end{figure}
\begin{figure}[htbp]
  \begin{minipage}[b]{0.45\linewidth}
    \centering
    \includegraphics[keepaspectratio, scale=0.5]{RLC_Parallel_Arg.png}
    \caption{RLC並列回路の偏角}
    \label{fig:RLC並列_偏角}
  \end{minipage}
  \begin{minipage}[b]{0.45\linewidth}
    \centering
    \includegraphics[keepaspectratio, scale=0.6]{RLC_Parallel_Locus.png}
    \caption{RLC並列回路のインピーダンスの軌跡}
    \label{fig:RLC_軌跡2}
  \end{minipage}
\end{figure}
\newpage

\section{考察}
\subsection{RLC直列回路の考察}
\subsubsection{特定の周波数(共振周波数)の理論値を求めよ。}
\label{sec:直列共振}
\begin{align}
  f_0 & = \frac{1}{2 \pi  \sqrt{LC}}                                            \\
      & = \frac{1}{2 \cdot \pi \sqrt{(100 \times 10^{-3})(0.1 \times 10^{-6})}} \\
      & = 1591.55 \,[\mathrm{Hz}]
\end{align}
\subsubsection{\ref{fig:RLC直列}の定数Rを変更したときにインピーダンスの値がどうなるか説明せよ。}
回路が共振時のインピーダンスの下限値が変わる
\label{sec:直列R}
\subsection{RLC並列回路の考察}
\subsubsection{特定の周波数(共振周波数)の理論値を求めよ。}
\begin{align}
  f_0 & = \frac{1}{2 \pi  \sqrt{LC}}                                            \\
      & = \frac{1}{2 \cdot \pi \sqrt{(100 \times 10^{-6})(0.1 \times 47^{-6})}} \\
      & = 7341.27 \,[\mathrm{Hz}]
\end{align}
\label{sec:並列共振}
\subsubsection{\ref{fig:RLC並列}の定数Rを変更したときにインピーダンスの値がどうなるか説明せよ。}
回路が共振時のインピーダンスの上限値が変わる
\label{sec:並列R}
\subsection{課題の答えと狙い}
共振する現象を原理から理解することで、実際に回路を組んで応用するときに有効である。
定数の変更などは自分の思い通りに設計する第一歩となりえる。


\section{結論}

\end{document}

