\documentclass[11pt,dvipdfmx]{jarticle}

\usepackage{eee}
\usepackage{subfig}
\usepackage{here}
\usepackage{url}
\renewcommand{\labelenumi}{\alph{enumi}}
\renewcommand{\labelenumii}{\roman{enumii}}
\newcounter{mycounter} % カウンタの宣言
\setcounter{mycounter}{0} % カウンタの初期化
\newcommand{\useMycounter}[1][]{\refstepcounter{mycounter}{#1}プログラム{\themycounter}: }

\begin{document}
% トップページを書く
\begin{jikkenTitle}
	\gakunen{3} % 学年を記述。この行で全体の枠を表示
	\numTitle{1}{電気電子計測} % 実験番号、タイトルを記述
	\subTitle{} % サブタイトルがあれば記述
	\jikkenbi{令和4年6月3日(木)} % 実験日を記述
	\jikkenbiII{令和4年6月10日(木)} % 実験日を記述(二日目がある場合。ない場合はこの行をコメントアウト)
	\kyoudou{} % 共同実験者名を記述
	\kyoudouII{} % その他の共同実験者名を記述
	\yoteibi{/  }% 予定日を記述
	\yoteibiII{}% 予定日2を記述
	\yoteibiIII{}% 予定日3を記述
	\hanNumberName{}{3333}{宮崎 永} % 班番号・学生番号・氏名を記述。この行でタイトルページの描画を終了
\end{jikkenTitle}

\section{目的}
本実験では
\begin{itemize}
	\item LabVIEWとMyRIOを使用して、素子の電圧電流特性について自動計測の方法を習得する。
	\item 測定データから近似直線式の傾き、切片を求める計算方法を習得する。
	\item 電圧電流特性から抵抗値を求める方法について習得する。
\end{itemize}
ことを目的とする。

\section{原理}
\subsection{LabVIEW}
LabVIEWは、各種計測器やmyRIOなどを用いて自動計測や制御を実装するためのグラフィカルユーザーインターフェイスのプログラミング言語である。
主な特徴は、ビルトインされた仮想計測器(以下VI)で、オシロスコープやマルチメーターなどの計測器と似た外観や機能をコンピューター上へ作成するというものである。
VIは、フロントパネル、ブロックダイアグラム、アイコン-コネクタという3つ主要素から構成される。
プログラミングは、ブロックダイアグラム上にアイコンを配置し、各アイコン間のコネクタをつなぐ形で行う。

\subsection{myRIO}
myRIOは、デュアルコアのARM Cortex-A9 リアルプロセッサとカスタマイズ可能なXilinx FPGA・アナログプロセッサの駆動するプログラミング言語には、LabVIEWを用いる。
LabVIEWとmyRIOを用いることにより、制御、ロボット、メカトロニクス、組込などを容易に実現することができる。

\subsection{myRIO ブレッドボードアクセサリ}
myRIOの拡張ポートに接続可能なブレッドボードアクセサリである。
myRIOの5V、3.3V、GND端子及びAnalog I/O、Digital I/Oの端子が、ブレッドボード上に結線した回路とヘッダにマッピングされている。
そのため、ブレッドボード上に結線した回路とヘッダとをジャンパ戦で結線することにより、回路への入出力制御および計測がmyRIOを用いて容易に実行することができる。

\subsection{真値と誤差及び相対誤差(誤差率)}
\subsubsection{真値}
真値とは、測定量(測定値ではない)が単位の何倍であるのかを示している値である\cite{keisoku}。
真値は必ず存在すると仮定しても我々は真値そのものは知ることができず、ただその存在する範囲を推定することが出来るだけである\cite{keisoku}。

\subsubsection{誤差及び相対誤差}
誤差は\weq{gosa}で定義される\cite{keisoku}。
\begin{eqnarray}
	誤差 = 測定値 − 真値
	\label{eq:gosa}
\end{eqnarray}
また、相対誤差とは真値に対する誤差の比である。
但し真値は分からないので、通常は\weq{soutaigosa}のように誤差が小さいとして真値の代わりに測定値で割る\cite{keisoku}。
\begin{eqnarray}
	相対誤差 = \frac{誤差}{真値} \simeq \frac{誤差}{測定値}
	\label{eq:soutaigosa}
\end{eqnarray}

\subsection{統計処理(正規分布・平均値・標準偏差)}
\subsubsection{正規分布}
左右対称の釣鐘型に値が分布しているのを正規分布といい、山の頂点に平均値がくる\cite{keisoku}。

\subsubsection{平均値}
平均値とは$N$個全てのデータの総和を$N$個で割って得られる値で、\weq{heikinchi}で表すことができる\cite{keisoku}。
\begin{equation}
	\bar{y} = \frac{1}{N}\sum^N_{i = 1}y_i
	\label{eq:heikinchi}
\end{equation}

\subsubsection{標準偏差}
標準偏差とは平均値を基準に各測定量がどれほどのばらついているかを定量的に表す値で、\weq{hyoujunhensa}で表すことができる\cite{keisoku}。
\begin{equation}
	\sigma = \sqrt{\frac{1}{N - 1}\sum^N_{i = 1}(y_i - \bar{y})^2}
	\label{eq:hyoujunhensa}
\end{equation}

\subsection{近似直線(最小二乗法)}
%。
2つの測定データ$y, x$間に一次方程式の関係があるとし、
\begin{equation}
	y = ax + b
	\label{eq:aiu}
\end{equation}
の傾き$a$、切片$b$を測定データから尤もらしい値にすることを考える。
その際に、
\begin{eqnarray}
	E &=& \sum\limits_{i=1}^{N} \varepsilon^2_i \nonumber\\
	&=& \sum\limits_{i=1}^{N} \bigl( y_i - f(x_i)\bigr)^2\nonumber\\\
	&=& \sum\limits_{i=1}^{N} \bigl( y_i - (ax_i+b)\bigr)^2
	\label{eq:error}
\end{eqnarray}
を最小にする$a$、$b$を求める。
これを最小二乗法といい、誤差を伴う測定値の処理においてその誤差の二乗の和を最小にすることで、最も確からしい関係式を求める方法である\cite{keisoku}。
\begin{eqnarray}
	\frac{\partial}{\partial a}E(a,b) &=& 0\\
	\frac{\partial}{\partial b}E(a,b) &=& 0
\end{eqnarray}
から得られる方程式を、それぞれ$a$、$b$について解けば良く、それぞれの解を得るための方程式は次の2つを用いることになる\cite{keisoku}。
\begin{equation}
	a = \frac{\sum_{n=1}^{n}(x_i -\bar{x})(y_i-\bar{y})}{\sum_{n=1}^{n}(x_i-\bar{x})^2}
	\label{eq:saisyou1}
\end{equation}
\begin{equation}
	b = \bar{y}-\frac{\sum_{n=1}^{n}(x_i -\bar{x})(y_i-\bar{y})}{\sum_{n=1}^{n}(x_i-\bar{x})^2} \bar{x}
	\label{eq:saisyou2}
\end{equation}

\section{方法}
\subsection{使用器具}
今回の実験で使用した器具を表\ref{tab:使用器具}に示す。
\begin{table}[htbp]
	\caption{使用器具}
	\begin{tabular}{cccc}
		\hline
		機器名                            & 製造元                  & 型番           & 固有番号                \\ \hline \hline
		PC                             & iiyama               & NK50SZ       & NKNK50SZ0000K000388 \\	\hline
		LabVIEW                        & National Instruments & ver19.0.1f3  &                     \\	\hline
		myRIO                          & National Instruments & myRIO-1900   & 309C570             \\ \hline
		NI myRIO Starter Accessory Kit & DIGILENT             & 6002-240-000 &                     \\ \hline
	\end{tabular}
	\label{tab:使用器具}
\end{table}

\subsection{実験手順}
\subsubsection{文字列および数値の入力・表示}
\begin{enumerate}[1)]
	\item	フロントパネルに数値制御器と数値表示器のブロックを作成した。
	\item 数値表示器と数値表示器をワイヤーで接続した。
	\item 数値制御器に任意の数値を入力した。
	\item 数値制御機に入力した数値が数値表示器に表示されていることを確認した。
\end{enumerate}

\subsubsection{数値計算(演算子・制御表示器作成・ワイヤ分岐・ブロック置換)}
\subsubsection*{足し算の作成}
\begin{enumerate}[1)]
	\item	フロントパネルに数値制御器ブロック2つと数値表示器ブロックを作成した。
	\item ブロックダイヤグラム上に和ブロックを配置した。
	\item 2の数値制御器を和ブロックの入力に、数値表示器の入力を和ブロックの出力にワイヤーで接続した。
	\item 2つの数値制御器に数値を入力し、数値表示器に二つの数字の和が表示されていることを確認した。
\end{enumerate}

\subsubsection*{コネクタを使用した足し算の作成}
\begin{enumerate}[1)]
	\item 和ブロックを配置した。
	\item 和ブロックのxコネクタを右クリックし、プルダウンメニューの作成から制御器を選択した。
	\item 同様に、yのコネクタを右クリックし制御器を配置し、x+yコネクタを右クリックし、表示器を配置した。
	\item 2つの制御器に数値を入力し、数値表示器に二つの数字の和が表示されていることを確認した。
\end{enumerate}

\subsubsection*{ワイヤ分岐による掛け算の作成}
\begin{enumerate}[1)]
	\item これまで作成したプログラムはそのままに、ブロックダイヤグラム上に積ブロックを配置した。
	\item 積ブロックのxコネクタと数値制御器xから出ているワイヤを接続した。
	\item 積ブロックのyコネクタと数値制御器yから出ているワイヤを接続した。
	\item 積ブロック右側のx*yコネクタを右クリックし、プルダウンメニューの作成から表示器を選択した。
	\item 制御器の数値を変えたときでも、表示器に表示される値が正しいことを確認した。
\end{enumerate}

\subsubsection*{引き算・割り算の作成(ブロックの置換)}
\begin{enumerate}[1)]
	\item 和ブロックを右クリックし、プルダウンメニューの置換から数値パレットで差を選択した。また、差ブロックの表示器の名前をx-yに変更した。
	\item 同様に、積ブロックを右クリックし、プルダウンメニューの置換から数値パレットで商を選択した。また、差ブロックの表示器の名前のx/yに変更した。
	\item	プログラムを実行し、それぞれの表示器に正しい値を表示されていることを確認した。
\end{enumerate}

\subsubsection{forループの作成}
\begin{enumerate}[1)]
	\item ブロックダイヤグラム上にforループブロックを作成した。
	\item 繰り返し回数Nブロックに繰り返し回数の定数を設定した。
	\item 和、数値制御器、数値制御器をforループ内に1つづつ用意した。
	\item 和ブロックのコネクタxとカウンタ変数iのブロックと繋いだ。
	\item プログラムを実行し、表示器の値を確認する。
	\item メニューバーの電球アイコンを選択し、カウンタ変数iの値が変化しながら実行されることを確認した。
	\item 数値制御器yのブロックをforループの外に配置した時の動作も確認した。
\end{enumerate}

\subsubsection{実験2-1:アナログ電圧値の計測}
\begin{enumerate}[1)]
	\item ジャンパ線を用いて、ブレッドボードに図\ref{fig:jikken1}に示す配線をした。
	      \begin{figure}[H]
		      \centering
		      \includegraphics[keepaspectratio, scale=1.0]{jikken1.pdf}
		      \caption{電圧値計測実験の配線図}
		      \label{fig:jikken1}
	      \end{figure}
	\item myRIOのAポートにブレッドボードを差し込み、次のように実験を行った。
	\item forループなどを用いて、AIO端子の電圧を100回測定できるようなプログラムをLabVIEWを用いて作成し、5V, 3.3V, GNDの出力電圧をそれぞれ計測した。また、電圧の切り替えはジャンパ線の接続先を変えることによって行った。
	\item 出力電圧が$0\,\mathrm{V}$から$5\,\mathrm{V}$まで$0.2\,\mathrm{V}$刻みで変化するようにプログラムを変更した。
	\item 測定したデータからexcelを用いて平均値と標準偏差を求めた。
\end{enumerate}

\subsubsection{実験2-2:電圧の出力}
\begin{enumerate}[1)]
	\item ジャンパ線を用いて、ブレッドボードに図\ref{fig:jikken2}に示す配線をした。
	      \begin{figure}[H]
		      \centering
		      \includegraphics[keepaspectratio, scale=1.0]{jikken2.pdf}
		      \caption{電圧の出力実験の配線図}
		      \label{fig:jikken2}
	      \end{figure}
	\item myRIOのAポートにブレッドボードを差し込み、次のように実験を行った。
	\item $0\,\mathrm{V}$から$5\,\mathrm{V}$まで0.05秒ごとに$0.5\,\mathrm{V}$刻みで出力電圧を変化させ、その電圧をAIO端子で測定するプログラムをLabVIEWを用いて作成し、実行した。
	\item	測定したデータから、Excelを用いて二乗平均平方根誤差を求めた。
\end{enumerate}

\subsubsection{実験3:素子の電圧電流特性の計測}
固定抵抗、可変抵抗、CdSセンサ、力センサ、発光ダイオードにおいて電圧電流特性の計測を行った。
すべての素子において実験手順は同様であり、次に示す。
\begin{enumerate}[1)]
	\item ジャンパ線を用いて、ブレッドボードに図\ref{fig:jikken3}に示す配線をした。$R_0$は電流計測用のシャント抵抗、$R_{\mathrm{X}}$は計測対象の素子である。
	      \begin{figure}[H]
		      \centering
		      \includegraphics[keepaspectratio, scale=1.0]{jikken3.pdf}
		      \caption{電圧の出力実験の配線図}
		      \label{fig:jikken3}
	      \end{figure}
	\item myRIOのAポートにブレッドボードを差し込んだ。
	\item $0\,\mathrm{V}$から$5\,\mathrm{V}$まで0.05秒ごとに$0.5\,\mathrm{V}$刻みで出力電圧AO0を変化させ、AI0とAI1それぞれの端子の電圧を測定するプログラムをLabVIEWを用いて作成し実行した。また、LabVIEW上で抵抗$R_0$の値から電流を求めるプログラムを記述した。
	\item	測定したデータから、Excelを用いて各素子の電圧電流特性のグラフを作成した。
\end{enumerate}

\subsubsection{実験3-1:固定抵抗の電圧電流特性の計測}
\begin{enumerate}[i)]
	\item 図\ref{fig:jikken3}の$R_\mathrm{X}$を$1\,\mathrm{k}\Omega$の固定抵抗に変更した。
	\item 図\ref{fig:jikken3}の$R_0$を$100\,\Omega$、$1\,\mathrm{k}\Omega$、$10\,\mathrm{k}\Omega$、$100\,\mathrm{k}\Omega$の固定抵抗にそれぞれ変更し、各素子について電圧電流特性を計測した。
\end{enumerate}

\subsubsection{実験3-2:可変抵抗の電圧電流特性の計測}
\begin{enumerate}[i)]
	\item $R_X$を図\ref{fig:teiko}に示す可変抵抗に変更した。このとき、3端子のうち1-2端子間を使用した。
	      \begin{figure}[H]
		      \centering
		      \includegraphics[keepaspectratio, scale=0.8]{teiko.pdf}
		      \caption{可変抵抗}
		      \label{fig:teiko}
	      \end{figure}
	\item 図\ref{fig:jikken3}の$R_0$を$100\,\Omega$の抵抗に変更した。
	\item 電流特性の計測時、図\ref{fig:teiko}の可変抵抗のボリュームのA、B、Cの各箇所において計測した。
	\item 同様の計測をほかの2-3端子間、1-3端子間でも行った。
\end{enumerate}

\subsubsection{実験3-3:CdSセンサの電圧電流特性}
\begin{enumerate}[i)]
	\item 図\ref{fig:jikken3}の$R_\mathrm{X}$をCdSセンサに変更した。
	\item 図\ref{fig:jikken3}の$R_0$を$100\,\Omega$の抵抗に変更した。
	\item センサーに懐中電灯を当てた明るい場合と、センサーを手で覆った暗い場合の両方で計測を行った。
\end{enumerate}

\subsubsection{実験3-4力センサの電圧電流特性}
\begin{enumerate}[i)]
	\item 図\ref{fig:jikken3}の$R_\mathrm{X}$を力センサーに変更した。
	\item 図\ref{fig:jikken3}の$R_0$を$100\,\Omega$の抵抗に変更した。
	\item センサーに力を加えていない状態と加えた状態の両方で計測を行った。
\end{enumerate}

\subsubsection{実験3-5発光ダイオードの電圧電流特性}
\begin{enumerate}[i)]
	\item 図\ref{fig:jikken3}の$R_\mathrm{X}$を発光ダイオードに変更した。このとき、素子の極性に注意しなければならない。カソードがGND、アノードがAO0側になるように接続した。
	\item 図\ref{fig:jikken3}の$R_0$を$100\,\Omega$の抵抗に変更した。
	\item センサーに力を加えていない状態と加えた状態の両方で計測を行った。
	\item 白、赤、緑、青色の発光ダイオードについて、それぞれ電圧電流特性を計測した。
\end{enumerate}
\newpage

\section{結果}
\subsection{実験2-1}
実験2-1で作成したプログラムをプログラム\ref{prog1}に、プログラム\ref{prog1}実行後の出力結果を\ref{kekka1}に示す。
また、出力結果から平均値・標準偏差を求め整理したものを表\ref{heikin-hensa}に示す。
5{[}V{]}と、GNDのの測定値と、3.3{[}V{]}の測定値を比較したとき、前者が後者に比べ誤差はかなり小さくなった。
\begin{figure}[H]
	\centering
	\includegraphics[keepaspectratio, scale=0.6]{ex21_2.pdf}\\
	\useMycounter[\label{prog1}]実験2-1のフローチャート
\end{figure}
\begin{figure}[H]
	\centering
	\includegraphics[keepaspectratio, scale=0.6]{ex21_1.pdf}\\
	\caption{実験2-1のフロントパネル}
	\label{kekka1}
\end{figure}
\begin{table}[H]
	\centering
	\caption{実験2-1の結果}
	\begin{tabular}{lll}
		測定した電圧{[}V{]} & 平均値{[}V{]} & 標準偏差{[}V{]}             \\ \hline \hline
		GND           & 0.004883   & 1.2143$\times 10^{-17}$ \\	\hline
		3.3           & 3.26296342 & 0.00017094              \\ \hline
		5             & 4.998779   & 9.77$\times 10^{-15}$   \\ \hline
	\end{tabular}
	\label{heikin-hensa}
\end{table}

\newpage
\subsubsection{実験2-2}
実験3-2で作成したプログラムをプログラム\ref{prog2}に、プログラム\ref{prog1}実行後の出力結果を\ref{kekka2}に示す。
また、出力結果より二乗平均平方誤差を求め整理したものを表\ref{heiho}に示す。
それぞれの出力電圧と測定電圧の誤差はかなり小さい値をとっているため、二乗平均平方誤差もかなり小さい値となっていることがわかる。
プログラム内のAnalog OutputとAnalog Inputの間にある遅延時間は、myRIOの出力電圧の立ち上がり時間を考慮したものである。この遅延時間を挿入しなかった場合、正しい電圧が計測できなかった。
\begin{figure}[H]
	\centering
	\includegraphics[keepaspectratio, scale=0.5]{ex22_2.pdf}\\
	\useMycounter[\label{prog2}]実験2-2のフローチャート
\end{figure}
\begin{figure}[H]
	\centering
	\includegraphics[keepaspectratio, scale=0.5]{ex21_1.pdf}\\
	\caption{実験2-2のフロントパネル}
	\label{kekka2}
\end{figure}
\begin{table}[H]
	\centering
	\caption{実験2-2の結果}
	\begin{tabular}{lcc}
		出力電圧{[}V{]}                          & 測定電圧{[}V{]} & 出力電圧と測定電圧の差{[}V{]} \\ \hline \hline
		0                                    & 0.007324    & -0.00732           \\	\hline
		0.5                                  & 0.499268    & 0.000732           \\	\hline
		1                                    & 0.999756    & 0.000244           \\ \hline
		1.5                                  & 1.501465    & -0.00147           \\	\hline
		2                                    & 1.998291    & 0.001709           \\	\hline
		2.5                                  & 2.50122     & -0.00122           \\	\hline
		3                                    & 3.001709    & -0.00171           \\	\hline
		3.5                                  & 3.499756    & 0.000244           \\	\hline
		4                                    & 4.002685    & -0.00269           \\	\hline
		4.5                                  & 4.499511    & 0.000489           \\	\hline
		5                                    & 4.998779    & 0.001221           \\	\hline	\hline
		\multicolumn{2}{c}{二乗平均平方根誤差{[}V{]}} & 0.002571                         \\ \hline
	\end{tabular}
	\label{heiho}
\end{table}

\subsection{実験3}
実験3で作成したプログラムをプログラム\ref{prog3}に、プログラム\ref{prog3}実行後の出力結果を\ref{kekka3}に示す。
この結果は実験3-2、可変抵抗の1-2端子間でボリューム位置がCの時の結果である。
実験3は5種類の素子で実験を行うが、すべてこのプログラムを使用して計測を行った。
\begin{figure}[H]
	\centering
	\includegraphics[keepaspectratio, scale=0.7]{ex31_2.pdf}\\
	\useMycounter[\label{prog3}]実験3のプログラム
\end{figure}
\begin{figure}[H]
	\centering
	\includegraphics[keepaspectratio, scale=0.7]{ex31_1.pdf}\\
	\caption{実験3のフロントパネル}
	\label{kekka3}
\end{figure}

\subsubsection{実験3-1}
実験3-1を行い得られた固定抵抗の電圧電流特性を図\ref{fig:3-1}に示す。
測定対象の素子は変更せず、$R_0$のみ変更したため、グラフの傾きは変化していない。
$R_0$の素子値が大きくなるにつれ、電流の変化量が少なくなっていることもわかる。
\begin{figure}[H]
	\centering
	\includegraphics[keepaspectratio, scale=0.7]{3-1.pdf}
	\caption{固定抵抗の電圧電流特性}
	\label{fig:3-1}
\end{figure}

\subsubsection{実験3-2}
実験3-2を行い得られた可変抵抗の電圧電流特性を図\ref{fig:3-2_1}、図\ref{fig:3-2_2}、図\ref{fig:3-2_3}に示す。
それぞれ順に1-2端子間、2-3端子間、1-3端子間において計測した結果である。
\begin{figure}[H]
	\centering
	\includegraphics[keepaspectratio, scale=0.7]{3-2_1.pdf}
	\caption{可変抵抗の端子1-2間の電圧電流特性}
	\label{fig:3-2_1}
\end{figure}
\begin{figure}[H]
	\centering
	\includegraphics[keepaspectratio, scale=0.7]{3-2_2.pdf}
	\caption{可変抵抗の端子2-3間の電圧電流特性}
	\label{fig:3-2_2}
\end{figure}
\begin{figure}[H]
	\centering
	\includegraphics[keepaspectratio, scale=0.7]{3-2_3.pdf}
	\caption{可変抵抗の端子1-3間の電圧電流特性}
	\label{fig:3-2_3}
\end{figure}

\newpage
\subsubsection{実験3-3}
実験3-3を行い得られたCdSセンサの電圧電流特性を図\ref{fig:3-3}に示す。
明るい場合は電圧が0.5V付近で電流が約7mA流れているのに対し、暗い場合はほとんど電流が流れていないる。
\begin{figure}[H]
	\centering
	\includegraphics[keepaspectratio, scale=0.7]{3-3.pdf}
	\caption{CdSセンサの電圧電流特性}
	\label{fig:3-3}
\end{figure}

\subsubsection{実験3-4}
実験3-4を行い得られた力センサの電圧電流特性を図\ref{fig:3-4}に示す。
力を加えていないときは、傾きがほぼ0であり、電流がほとんど変化していなかった。
力を加えた場合、5V付近で約1.4mAほど流れており、力を加えていないときに比べて電流を流していた。
\begin{figure}[H]
	\centering
	\includegraphics[keepaspectratio, scale=0.7]{3-4.pdf}
	\caption{力センサの電圧電流特性}
	\label{fig:3-4}
\end{figure}

\newpage
\subsubsection{実験3-5}
実験3-5を行い得られたLEDの電圧電流特性を図\ref{fig:3-5}に示す。
単色LEDの発光色の波長が短いほど順方向降下電圧$V_\mathrm{F}$が大きくなった。
白色LEDは赤色より$V_\mathrm{F}$が大きいが緑色LEDよりは小さくなった。

\begin{figure}[H]
	\centering
	\includegraphics[keepaspectratio, scale=0.7]{3-5.pdf}
	\caption{LEDの電圧電流特性}
	\label{fig:3-5}
\end{figure}

\section{考察}
\subsection{理想の平均値及び標準偏差}
平均値は原理の項で示した通り、N個のデータの総和をNで割って得られる値である。
測定値に誤差がなければN個すべてのデータが同じであるので、平均値はN個のデータそれぞれの値と等しくなる。
これはどのデータも他のデータと比べた時に差がないことを意味する。
平均偏差は原理の項で示した通り、平均値を基準としたときに各データがどれぐらいのばらつきがあるのかを表す値である。
測定したデータがすべて平均値と等しいと仮定したとき、平均値と各測定値のばらつきはない。
よって、理想の平均値はあるデータ群のうちの一つのデータと等しい値で、理想の標準偏差は0であると考える。

\subsection{出力電圧値と計測電圧値の関係}
\begin{figure}[H]
	\centering
	\includegraphics[keepaspectratio, scale=0.5]{5-2.pdf}
	\caption{出力電圧と入力電圧の関係}
	\label{fig:5-2}
\end{figure}
図\ref{fig:5-2}は実験2-1で得られたデータをもとに、グラフとして描画したものである。
図\ref{fig:5-2}はほぼ傾きが1で、ほぼ原点を通る比例であることがわかる。
理想的には出力電圧と入力電圧が等しくなると考えられる。よって傾き1で切片0の$y=x+0$となる。
しかし実際には傾きがわずかに小さく、正方向のごく小さな切片が存在していることがわかる。
微小な切片が存在する原因として、出力電圧が0Vのときに入力電圧がわずかに存在していたためだと考えることができる。
また、傾きが1よりもわずかに小さい理由として、ジャンパ線やブレッドボードの抵抗成分が原因であると考えた。
よって図\ref{fig:5-2}のグラフは理想の式より傾きが小さく、正方向に切片が存在しているのだと考えられる。

\subsection{誤差の変化}
電圧・電流・抵抗の値をオームの法則で表すと次のようになる。
\begin{equation}
	V = IR
	\label{eq:ohm}
\end{equation}
式(\ref{eq:ohm})を変形し、電圧電流特性の傾きを$a$としたときに次のように表せる。
\begin{align}
	a & = \frac{1}{R\, [\Omega]}
	\label{eq:henkei1}
\end{align}
式(\ref{eq:henkei1})を用いて傾きを求め、その値から抵抗値の計算値と誤差を導出し整理したものを表\ref{tab:gosa}に表す。
\begin{table}[htbp]
	\centering
	\caption{抵抗の誤差率}
	\begin{tabular}{c|ccc}
		$抵抗値\, [\Omega]$ & $傾き\, [\mathrm{S}]$ & $R_\mathrm{X}\, [\mathrm{k} \Omega]$ & $誤差\, [\mathrm{k} \Omega]$ \\	\hline
		100              & 1.095635            & 0.912713                             & -0.087287                  \\
		1k               & 1.096803            & 0.911741                             & -0.088259                  \\
		10k              & 1.097618            & 0.911064                             & -0.088936                  \\
		100k             & 1.097477            & 0.911181                             & -0.088819                  \\
	\end{tabular}
	\label{tab:gosa}
\end{table}
この表より、最も誤差が少ないのは$100\,\Omega$のときであることがわかる。$100\,\Omega$の場合のみ誤差が小さいので、
ブレッドボードにうまく差し込めていなかったか、抵抗のリード線の表面が酸化していたなどの理由が考えられる。
\subsection{可変抵抗器の内部構造}
実験3-2より、1-2端子間ではAの位置、Bの位置、Cの位置の電流を比較すると、C>B>Aの順で電流が多く流れていることがわかる。
同様に、2-3端子間で電流を比較すると、A>B>Cの順で電流が多く流れている。また、1-3端子間ではつまみがどの位置でも電流がほぼ同じである。これはどのつまみの位置でも抵抗値が等しいことを表している。
このことから予測した、可変抵抗の内部構造を図\ref{fig:r}に示す。
\begin{figure}[H]
	\centering
	\includegraphics[keepaspectratio, scale=1.0]{r.pdf}
	\caption{可変抵抗の内部構造}
	\label{fig:r}
\end{figure}

\subsection{CdSセンサの抵抗値と光強度の関係}
CdSセンサは、カドミウムと硫黄の化合物で構成されている。カドミウムの最外殻電子は、原子核との結合が弱く、自由電子となりやすい\cite{cds}。
これに硫黄を結合させると、カドミウムの自由電子を硫黄原子が捕捉して絶縁体に変化する\cite{cds}。
しかし、硫黄の自由電子の捕捉力は弱く、光が当て、エネルギーを与えると自由電子を放出して導体に変化する\cite{cds}。
よって光の量によって放出する自由電子の量が変化し、抵抗値が変化することになる。
実験3-3で得られた結果より、光をあてたときに電圧の変化に対して電流の変化が大きく、光をあてていないときは電圧の変化に対し電流の変化が小さいことがわかる。
以上のことより、実験結果が正しいことがわかる。

\subsection{力センサの抵抗値と加える力の強さの関係}
力センサは金属に力を加えたときのひずみによる抵抗値の変化を利用した素子である。力を加えると、センサ内部の導体の断面積が増加して抵抗値が減少し、力を加えないと導体の断面積が減少して抵抗値が増加する\cite{power}。
実験3-4で得られた結果より、センサに力を加えているときは電圧と電流が比例の関係にあることがわかる。
一方で力を加えていないときは電圧を変化させても電流が増加せず、ほぼ0であった。
以上のことより、実験結果が正しいことがわかる。

\subsection{LEDの発光色と電圧電流特性の関係}
本実験で使用したLEDの波長の長さは、短いほうから順に青、緑、赤である。白色は可視光を含むすべての光の波長がまんべんなく含まれている。
各色のLEDを点灯させるのに必要な電圧は、青色が約3.6V、緑色が約2.1V、赤色が1.8V、白色(青色LED+黄色蛍光体)が約3.6Vである。
光(波)を生じさせるためにはエネルギーを与える必要がある。このエネルギー量は次の式で求まる。
\begin{equation}
	E = \frac{2 \pi^2 m c^2 A^2}{\lambda^2}
	\label{eq:hikari}
\end{equation}
式(\ref{eq:hikari})より、波長の短い光を生み出す為にはより大きなエネルギーが必要になることがわかる。
ダイオードの特性として、不感領域の間は抵抗値が大きく、不感領域を超えると急激に電流が流れやすくなる。
この電流が急激に流れやすくなる電圧の閾値は発光色の波長によって決められると考えられる。
この電圧は式(\ref{eq:hikari})より、発光色の波長と反比例にあることがわかる。

\subsection{可変抵抗の抵抗値}
式(\ref{eq:henkei1})で示したように、電圧電流特性の傾きは抵抗値に依存している。
実験3-2で得られた結果より、最も抵抗値が大きいのは、最も傾きが小さい1-3端子間のときである。
また、最も抵抗値が小さいのは、最も傾きが大きい1-3端子間のつまみ位置C、2-3端子間のつまみ位置Cである。
電圧$0\,\mathrm{V}$のときに電流が振り切っているので、ほぼ抵抗値は$0\,\Omega$であることがわかる。
次に示す式と図\ref{fig:3-2_3}を用いて抵抗の最大値は次のように求まった。
\begin{align}
	I & = \frac{V}{R}                        \\
	R & = \frac{V}{I}                        \\
	  & \simeq \frac{1}{0.10 \times 10^{-3}} \\
	  & \simeq 10\,[\mathrm{k}\Omega]
\end{align}
よって、可変抵抗の抵抗値がとる値の範囲は、$0\,\Omega \leq R \leq 10\,\mathrm{k}\Omega$であると考えられる。

\section{結論}
本実験では、
\begin{itemize}
	\item LabVIEWとMyRIOを使用して、素子の電圧電流特性について自動計測の方法を習得する。
	\item 測定データから近似直線式の傾き、切片を求める計算方法を習得する。
	\item 電圧電流特性から抵抗値を求める方法について習得する。
\end{itemize}
という目的の下、実験を行った。
実験でmyRIOを使用して計測を行い、得られたデータよりExcelを用いて近似直線式の傾きと切片を求めることができた。また、考察では電圧電流特性より抵抗値を算出することができた。
よって本実験の目的はすべて達成することができた。

\begin{thebibliography}{9}%参考文献数が10以上の場合は9を99に変更
	%\bibitem{xxx}の引用を本文中で行うには\cite{xxx}と記述。
	\bibitem{keisoku} 阿部武雄/村山実, 電気・電子計測第4版,森北出版,2019.
	\bibitem{cds} アイアール技術者教育研究所, ``3分でわかる技術の超キホン CdSセルとは?原理と電子回路での使い方を解説!'' \url{https://engineer-education.com/cds-cell/} ,参照 Apr. 6,2022.
	\bibitem{power} 共和電業株式会社, ``ひずみゲージ入門'' \url{https://www.kyowa-ei.com/jpn/technical/strainbasic_course/index.html} ,参照 Apr. 6,2022.
\end{thebibliography}


\end{document}