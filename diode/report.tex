\documentclass[11pt,dvipdfmx]{jarticle}
\usepackage{eee}
\usepackage{subfig}
\usepackage{here}
\usepackage{url}

\begin{document}
% トップページを書く
\begin{jikkenTitle}
	\gakunen{3} % 学年を記述。この行で全体の枠を表示
	\numTitle{2}{ダイオードの特性} % 実験番号、タイトルを記述
	\subTitle{} % サブタイトルがあれば記述
	\jikkenbi{令和4年6月23日(木)} % 実験日を記述
	%\jikkenbiII{平成28年7月6日(木)} % 実験日を記述(二日目がある場合。ない場合はこの行をコメントアウト)
	\kyoudou{3325 谷下 文紀 3329 野島 奏一朗 3337吉野 曹生} % 共同実験者名を記述
	\kyoudouII{} % その他の共同実験者名を記述
	\yoteibi{}% 予定日を記述
	\yoteibiII{}% 予定日2を記述
	\yoteibiIII{}% 予定日3を記述
	\hanNumberName{4}{3333}{宮崎 永} % 班番号・学生番号・氏名を記述。この行でタイトルページの描画を終了
\end{jikkenTitle}

\section{目的}
今回の実験では以下の2点を目的とする。
\begin{itemize}
	\item ダイオードの原理を知り,実験により整流作用を理解することでダイオードを使用できるようにする。
	\item ツェナーダイオードの特性を学び,ツェナーダイオードを使用できるようにする。
	\item 太陽電池の特性を知り,使用できるようにする。
\end{itemize}

\section{原理}
\subsection{ダイオード}
半導体内部には,電子と正孔がキャリアとして存在している。
真性半導体には4価のシリコンがよく使われる。
真性半導体に微量の不純物を混入させたものを不純物半導体と言い,不純物としてリンやヒ素のような5価の元素(ドナー)を用いたものをn型半導体,ホウ素やガリウムのような3価の元素(アクセプタ)を用いたものをp型半導体と呼ぶ。

\wfig{fig1}のように,p型半導体とn型半導体を接合し,端子を付けたものをダイオードと呼ぶ。
ダイオード内部において,正孔はp型半導体内では多数キャリア,n型半導体内では少数キャリアであるから,より密度の大きいp型領域からn型領域へ流れ込む。
この現象を拡散と呼ぶ。
また,n領域へ拡散した正孔はn領域内の電子と結合し,双方とも消滅する。
したがって,n領域では正に帯電したドナーイオンが,p領域では負に帯電したアクセプタイオンのみが残り,平衡状態となる。
この結果,pn接合近傍にはキャリアの存在しない空乏層が形成される。
空乏層では,電化分布によりp領域側からn領域側へ電位差が発生し<これを拡散電位と呼ぶ。

\begin{figure}[!h]
	\centering
	\includegraphics[width=8.2cm]{fig1.eps}
	\caption{pn接合と空乏層}
	\label{fig:fig1}
\end{figure}%

\wfig{fig2}のようにp型半導体に正,n型半導体に負の電圧を加えると,p領域とn領域の電位差は拡散電圧+印加電圧となり,電圧差が減少するため拡散電位により阻止されていたキャリアの拡散が起こる。
このとき,ダイオードに印加した電圧,流れた電流を,それぞれ順方向電圧,順方向電流という。
逆に,n型半導体に正,p型半導体に負の電圧を加えると,p領域とn領域の電圧差が大きくなるため電流は殆ど流れなくなる。
このとき,ダイオードに加えた電圧,流れた電流を,それぞれ逆方向電圧,逆方向電流という。
また,この電圧のかけ方を逆バイアスという。

\begin{figure}[!t]
	\centering
	\includegraphics[height=8.0cm]{fig2.eps}
	\caption{ダイオードの動作原理}
	\label{fig:fig2}
\end{figure}%

\subsection{ツェナーダイオード}
ツェナーダイオードは定電圧ダイオードとも呼ばれ,加わる電圧がある一定の値(ツェナー電圧)を上回ると,急激に電流が流れるようになる素子である。
このとき流れる電流をツェナー電流と呼ぶ。
入力電圧の増加に伴いツェナー電流が増加するため,出力電圧は一定に保たれる。
すなわち,ツェナーダイオードは端子間電圧がツェナー電圧以上ならON,ツェナー電圧以下ならOFFといったようにスイッチと似たような動作をして,ほぼ一定の電圧を維持する素子である。

\subsection{太陽電池}
太陽電池は様々な種類のものがあるが,本実験では最も基本的な構造である結晶シリコン太陽電池を用いる。
結晶シリコン太陽電池の構造は,\wfig{fig3}のような上段がn 形半導体,下段がp 形半導体となっており,pn結合している。
\wfig{fig4}で示すように,光が照射されると光のエネルギーにより接合面の電子はn形半導体へ,正孔はp形半導体へ移動し,起電力が発生する(光起電力効果)。
この起電力は,光が照射されている間は持続し外部に負荷を接続しておけば電力を供給することができる。
また,電子は負荷を通ってp形半導体に戻り,正孔と結合する。
\begin{figure}[htb]
	\centering
	\includegraphics[width=6.3cm]{fig3.eps}
	\caption{結晶シリコン太陽電池の構造}
	\label{fig:fig3}
\end{figure}

\begin{figure}[htb]
	\centering
	\includegraphics[width=12cm]{fig4.eps}
	\caption{太陽電池の動作原理}
	\label{fig:fig4}
\end{figure}

太陽電池の電流--電圧特性は\wfig{fig5}に示したようなダイオードの特性を下にシフトした特性となる。
ここで電流値は負の値になるが,正に消費と考えていたので負の値は発電していることを意味する。
一般的には,発電した電力も正の値で表現するので,縦軸を反転させる(\wfig{fig5}参照)。
また,太陽電池の電流--電圧特性をI-Vカーブ,電力--電圧特性をP-Vカーブと呼ぶ。
\wfig{fig6}に示すように,I-V カーブは第1象限のみを表示する。


ここで電圧が0\,Vになる(短絡する》 時の電流を短絡電流($I_{\mathrm{sc}}$),電流が0\,A になる電圧を解放電圧($V_{\mathrm{oc}}$)と呼ぶ。
また,縦軸を電力とした\wfig{fig6}に示すP-Vカーブからわかる通り,太陽電池はどのようなどのような電圧で利用しても同じ電力を得られるわけではない。
太陽電池を有効に活用するためには,発電電力が最大になる最大電力点で利用する必要がある。

ここで,発電電力が最大になる電力を最大電力($P_{\mathrm{max}}$),その時の電圧を最適動作電圧($V_{\mathrm{pm}}$) と呼ぶ。
$P_{\mathrm{max}}$は,気象条件によって大きく変化し$V_{\mathrm{pm}}$も変化してしまう。
そのため,太陽電池を利用したシステム(太陽光発電システム)では,常に$P_{\mathrm{max}}$ を追従する制御であるMPPT制御(Maximum Power Point Tracking:最大電力点追尾)が具備されている。
MPPTには様々な手法があるが,その手法を考える上でも太陽電池のI-Vカーブの測定は非常に重要である。

\begin{figure}[htb]
	\centering
	\includegraphics[width=8.5cm]{fig5.eps}
	\caption{ダイオードと太陽電池の電流--電圧特性}
	\label{fig:fig5}
\end{figure}
\begin{figure}[htb]
	\centering
	\includegraphics[width=10.3cm]{fig6.eps}
	\caption{太陽電池の発電特性(左:I--Vカーブ,右:P--Vカーブ)}
	\label{fig:fig6}
\end{figure}
\clearpage

\section{実験}
\subsection{ダイオードの特性}
\subsubsection{実験方法}
本実験では,以下に示す手順で実験を行った。
\begin{enumerate}[i)]
	\item	図\ref{fig:for}の回路を作成した。このとき抵抗は$5\,\Omega$,ダイオードは1N4002を用いた。$5\,\Omega$の抵抗が不足していたため,$10\,\Omega$の抵抗を2並列にして使用した。
	\item この回路に順バイアス$E$を加えた。このとき,$V_\mathrm{D}$を$0\,\mathrm{V}$から$0.8\, \mathrm{V}$まで$0.1\, \mathrm{V}$刻みで変化させ,それぞれ電圧$V_\mathrm{D}$と電流$I_\mathrm{D}$を計測した。
	\item 計測した値より,$V_\mathrm{D}—I_\mathrm{D}$特性のグラフを作成した。
	\item 図\ref{fig:rev}の回路を作成した。各受動素子の値は図\ref{fig:for}と同様である。
	\item この回路に順バイアス$E$を加えた。このとき,$V_\mathrm{D}$を$0\,\mathrm{V}$から$8\, \mathrm{V}$まで$1\, \mathrm{V}$刻みで変化させ,それぞれ電圧$V_\mathrm{D}$と電流$I_\mathrm{D}$を計測した。
	\item 計測した値より,$V_\mathrm{D}—I_\mathrm{D}$特性のグラフを作成した。
\end{enumerate}

\begin{figure}[H]
	\centering
	\includegraphics[keepaspectratio, scale=0.5]{For.pdf}
	\caption{ダイオード特性(順バイアス)測定回路}
	\label{fig:for}
\end{figure}
\begin{figure}[H]
	\centering
	\includegraphics[keepaspectratio, scale=0.5]{Rev.pdf}
	\caption{ダイオード特性(逆バイアス)測定回路}
	\label{fig:rev}
\end{figure}

\subsubsection{使用機器}
本実験で使用した使用機器を表\ref{tab:kiki1}に示す。
\begin{table}[H]
	\centering
	\caption{ダイオード特性実験の使用機器}
	\begin{tabular}{llll}
		\hline
		名称   & 製造元      & 型番                   & 製造番号/管理番号       \\ \hline
		電源装置 & YOKOGAWA & PA1811               & L96-000668      \\
		電圧計  & YOKOGAWA & TYPE 2011 CLASS 0.5  & B-5035.58.1/5   \\
		電流計  & YOKOGAWA & TYPE 2011 CLASS 0.5  & B-6051.45.6/8   \\
		電流計  & YOKOGAWA & MODEL 2011 CLASS 1.0 & B-5036.H1.10/10 \\ \hline
	\end{tabular}
	\label{tab:kiki1}
\end{table}

\subsubsection{結果}
\begin{enumerate}[i)]
	\item 順バイアス\\
	      計測した$V_\mathrm{D}$と$I_\mathrm{D}$を表\ref{tab:d1}に示す。また,この値をグラフに描画したものを図\ref{fig:d1}に示す。
	      \begin{figure}[H]
		      \centering
		      \includegraphics[keepaspectratio, scale=0.8]{D1.pdf}
		      \caption{ダイオード特性(順バイアス)}
		      \label{fig:d1}
	      \end{figure}
	      \begin{table}[H]
		      \centering
		      \caption{順バイアスの電圧と電流の関係}
		      \begin{tabular}{ll}
			      \hline
			      電圧$V_\mathrm{D}\,[\mathrm{V}]$ & 電流$I_\mathrm{D}\,[\mathrm{mA}]$ \\ \hline
			      0                              & 0.00                            \\
			      0.100                          & 0.00                            \\
			      0.200                          & 0.00                            \\
			      0.300                          & 0.00                            \\
			      0.400                          & 0.00                            \\
			      0.500                          & 0.10                            \\
			      0.550                          & 0.32                            \\
			      0.575                          & 0.73                            \\
			      0.600                          & 1.21                            \\
			      0.625                          & 2.03                            \\
			      0.650                          & 4.50                            \\
			      0.675                          & 7.20                            \\
			      0.700                          & 12.00                           \\
			      0.705                          & 13.20                           \\
			      0.710                          & 14.90                           \\
			      0.715                          & 16.20                           \\
			      0.720                          & 17.90                           \\
			      0.725                          & 19.50                           \\
			      0.730                          & 21.30                           \\
			      0.735                          & 23.50                           \\
			      0.740                          & 24.90                           \\
			      0.745                          & 27.80                           \\
			      0.750                          & 30.90                           \\
			      0.755                          & 31.90                           \\
			      0.760                          & 34.10                           \\
			      0.765                          & 37.40                           \\
			      0.770                          & 40.00                           \\
			      0.775                          & 43.20                           \\
			      0.800                          & 59.80                           \\	\hline
		      \end{tabular}
		      \label{tab:d1}
	      \end{table}
	      \ref{fig:d1}より,$V_\mathrm{D}$が$0.4\,\mathrm{V}$までは$I_\mathrm{D}$は$0\, \mathrm{mA}$であったが,$0.6\,\mathrm{V}$を超えると$I_\mathrm{D}$が急激に増加していることがわかる。

	\item 逆バイアス\\
	      計測した$V_\mathrm{D}$と$I_\mathrm{D}$を表\ref{tab:d2}に示す。また,この値をグラフに描画したものを図\ref{fig:d2}に示す。
	      \begin{figure}[H]
		      \centering
		      \includegraphics[keepaspectratio, scale=0.8]{D2.pdf}
		      \caption{ダイオード特性(逆バイアス)}
		      \label{fig:d2}
	      \end{figure}
	      \begin{table}[H]
		      \centering
		      \caption{逆バイアスの電圧と電流の関係}
		      \begin{tabular}{ll}
			      \hline
			      電圧$V_\mathrm{D}\,[\mathrm{V}]$ & 電流$I_\mathrm{D}\,[\mathrm{\mu A}]$ \\ \hline
			      0.0                            & 0.0                                \\
			      1.0                            & 0.0                                \\
			      2.0                            & 0.0                                \\
			      3.0                            & 0.0                                \\
			      4.0                            & 0.0                                \\
			      5.0                            & 0.0                                \\
			      6.0                            & 0.0                                \\
			      7.0                            & 0.0                                \\
			      8.0                            & 0.0                                \\ \hline
		      \end{tabular}
		      \label{tab:d2}
	      \end{table}
	      計測結果より,電圧$V_\mathrm{D}$が$8\,\mathrm{V}$までは電流$I_\mathrm{D}$が$0\,\mathrm{\mu A}$から変化しないことがわかる。
\end{enumerate}

\subsubsection{考察}
\begin{enumerate}[i)]
	\item 直流電圧電流特性を図\ref{fig:tokusei}に示す。この図は図\ref{fig:d2}に$E_\mathrm{Vp}$を$0.8\,\mathrm{V}$としたときの負荷線を加えたものである。
	      動作点$P$の微分抵抗$r_\mathrm{d}$は次の式で表される。
	      \begin{align}
		      \label{eq:teiko}
		      r_\mathrm{d} & = \frac{\Delta V_\mathrm{D}}{\Delta I_\mathrm{D}} \\ \nonumber
		                   & = \frac{0.72-0.715}{(17.90-16.20) \times 10^{-3}} \\	\nonumber
		                   & \approx 2.94\,[\Omega]                            \\ \nonumber
	      \end{align}

	      式(\ref{eq:teiko})より,動作点Pの微分抵抗$r_\mathrm{d}$は$2.94\,[\Omega]$であることがわかる。
	      \begin{figure}[H]
		      \centering
		      \includegraphics[keepaspectratio, scale=0.8]{tokusei.pdf}
		      \caption{直流電圧電流特性}
		      \label{fig:tokusei}
	      \end{figure}
	      $r_\mathrm{d}$,$V_\mathrm{j}$,理想ダイオードで構成される等価回路を図\ref{fig:rd}に示す。
	      \begin{figure}[H]
		      \centering
		      \includegraphics[keepaspectratio, scale=0.8]{rd.pdf}
		      \caption{ダイオードの等価回路}
		      \label{fig:rd}
	      \end{figure}
	      この等価回路に電圧源を加えたとき,ダイオードの内部電圧源$V_\mathrm{j}$よりも高かった場合,電流はダイオードに流れる。
	      一方で電圧源よりもダイオードの内部電圧源よりも小さい場合は,ダイオードに電流は流れない回路と考えられる。
	      また,抵抗$r_\mathrm{D}$は立ち上がり電圧からの電流の変化が緩やかになる要因とも考えることができる。

	\item 逆方向電圧値
	      本実験で使用したダイオードのデータシートを参照すると,$25\,\mathrm{^\circ C}$のときの逆電流は$0.05\,\mathrm{\mu A}$,$100\,\mathrm{^\circ C}$のときは$1.0\,\mathrm{\mu A}$であった\cite{d}。
	      ダイオードに逆バイアスを印加したときに,わずかに電流が流れるのは接合部付近の空乏層が原因であると考えられる。
	      逆バイアスをダイオードに印加すると,禁制帯が狭くなり,充満帯の電子がトンネル効果で伝導体に移動する現象が起きる。
	      よって逆方向電流は,トンネル効果による電子の移動によって生じる電流だと考えることができる。
\end{enumerate}
\newpage

\subsection{ツェナーダイオードの特性}
\subsubsection{実験方法}
本実験では,以下に示す手順で実験を行った。
\begin{enumerate}[i)]
	\item	図\ref{fig:for}の回路を作成した。このとき抵抗は$300\,\Omega$,ダイオードは1N4736を用いた
	\item この回路に順バイアス$E$を加えた。このとき,$V_\mathrm{D}$を$0\,\mathrm{V}$から$0.8\, \mathrm{V}$まで$0.1\, \mathrm{V}$刻みで変化させ,それぞれ電圧$V_\mathrm{D}$と電流$I_\mathrm{D}$を計測した。
	\item 計測した値より,$V_\mathrm{D}—I_\mathrm{D}$特性のグラフを作成した。
\end{enumerate}


\begin{figure}[H]
	\centering
	\includegraphics[keepaspectratio, scale=0.5]{zener.pdf}
	\caption{ツェナーダイオード特性測定回路}
	\label{fig:zener}
\end{figure}

\subsubsection{使用機器}
本実験で使用した使用機器を表\ref{tab:kiki2}に示す。
\begin{table}[H]
	\centering
	\caption{使用機器}
	\begin{tabular}{llll}
		\hline
		名称   & 製造元      & 型番                  & 製造番号/管理番号     \\ \hline
		電源装置 & YOKOGAWA & PA1811              & L96-000668    \\
		電圧計  & YOKOGAWA & TYPE 2011 CLASS 0.5 & B-5035.58.1/5 \\
		電圧計  & YOKOGAWA & TYPE 2051 CLASS 1.0 & B-3037.50.1/6 \\
		電流計  & YOKOGAWA & TYPE 2011 CLASS 0.5 & B-6051.45.6/8 \\ \hline
	\end{tabular}
	\label{tab:kiki2}
\end{table}

\subsubsection{結果}
計測した電圧$V_\mathrm{i}$と電流$I_\mathrm{Z}$を表\ref{tab:zd}に示す。また,この値をグラフに描画したものを図\ref{fig:zd}に示す。
\begin{figure}[H]
	\centering
	\includegraphics[keepaspectratio, scale=0.8]{ZD.pdf}
	\caption{ツェナーダイオード特性}
	\label{fig:zd}
\end{figure}
\begin{table}[H]
	\centering
	\caption{電源電圧$\,V_\mathrm{i}$, 電圧$\,V_\mathrm{L}$ と電流$\,I_\mathrm{Z}$の関係}
	\begin{tabular}{lll}
		\hline
		電圧$V_\mathrm{i}\,[\mathrm{V}]$ & 電流$I_\mathrm{Z}\,[\mathrm{mA}]$ & 電圧$V_\mathrm{L}\,[\mathrm{V}]$ \\ \hline
		0                              & 0.0                             & 0.00                           \\
		1                              & 0.0                             & 0.75                           \\
		2                              & 0.0                             & 1.50                           \\
		3                              & 0.0                             & 2.30                           \\
		4                              & 0.0                             & 3.00                           \\
		5                              & 0.0                             & 3.80                           \\
		6                              & 0.0                             & 4.50                           \\
		7                              & 0.0                             & 5.30                           \\
		8                              & 0.0                             & 6.05                           \\
		9                              & 0.0                             & 6.80                           \\
		10                             & 2.7                             & 7.00                           \\
		11                             & 5.9                             & 7.01                           \\
		12                             & 9.1                             & 7.05                           \\
		13                             & 12.5                            & 7.10                           \\
		14                             & 15.8                            & 7.11                           \\
		15                             & 19.2                            & 7.15                           \\
		16                             & 22.3                            & 7.19                           \\
		17                             & 25.7                            & 7.20                           \\
		18                             & 29.0                            & 7.25                           \\ \hline
	\end{tabular}
	\label{tab:zd}
\end{table}
計測結果より,電圧$V_\mathrm{L}$は$V_\mathrm{i}$が$9.0\,\mathrm{V}$付近までは急激に増加するが,$9.0\, \mathrm{V}$を超えると変化が小さくなっていることがわかる。
一方,電流$I_\mathrm{Z}$は$V_\mathrm{L}$が$9.0\, \mathrm{V}$付近までは$0\,\mathrm{mA}$であったが,$9.0\, \mathrm{V}$を超えると急激に増加した。
\subsubsection{考察}
\begin{enumerate}[i)]
	\item ツェナーダイオードの公称値と計測結果の比較
	      本実験で使用したツェナーダイオード1N4736Aは,立ち上がり電圧$V_\mathrm{Z}$は$6.8\,\mathrm{V}$で,公称誤差が$5\%$であった\cite{zd}。
	      実験の計測で得られた立ち上がり電圧は約$7.0\,\mathrm{V}$であり,データシートの公称値の範囲であることがわかる。
	      よって本実験は正しく計測が行えていたものと考えられる。
	\item ツェナーダイオードの使い方
	      ツェナーダイオードは電流が変化しても電圧が一定であるという特長を利用して定電圧回路に使用されたり,サージ電流や静電気からICなどを守る保護素子として使用される\cite{rohm}。
	\item $V_\mathrm{L} - I_\mathrm{Z}$特性
	      $V_\mathrm{L} - I_\mathrm{Z}特性$を図\ref{fig:zd2}に示す。電圧$V_\mathrm{L}$が約$6.8\,\mathrm{V}$のときに電流$I_\mathrm{Z}$が立ち上がっているため,ツェナー電圧$V_\mathrm{Z}$は$6.8\,\mathrm{V}$であることがわかる。
	      \begin{figure}[H]
		      \centering
		      \includegraphics[keepaspectratio, scale=0.4]{ZD2.pdf}
		      \caption{$V_\mathrm{L} - I_\mathrm{Z}特性$}
		      \label{fig:zd2}
	      \end{figure}
	\item	入出力特性の考察
	      \ref{fig:zd}より電圧$V_\mathrm{L}$が増加している区間ではツェナー電流$I_\mathrm{Z}$は増加していない。
	      一方で,ツェナー電流$I_\mathrm{Z}$が増加している区間では電圧$V_\mathrm{L}$は増加していない。
	      よってツェナーダイオードは,出力電圧とツェナー電圧は同時に増加しないような素子であると考えられる。


\end{enumerate}


\subsection{ツェナーダイオード定電圧回路}
\subsubsection{実験方法}
本実験では,以下に示す手順で実験を行った。
\begin{enumerate}[i)]
	\item	図\ref{fig:for}の回路を作成した。このとき抵抗$R$は$300\,\Omega$,抵抗$R_\mathrm{L}$は可変抵抗,ダイオードは1N4736を用いた
	\item この回路に電圧$V_\mathrm{i}=15\,\mathrm{V}$を印加した。このとき,可変抵抗$R_\mathrm{L}$を変化させ,$2\,\mathrm{mA}$から$22\, \mathrm{mA}$まで$0.1\, \mathrm{mA}$刻みで変化させた。それぞれ電圧$V_\mathrm{D}$と電流$I_\mathrm{D}$を計測した。
	\item 計測した値より,$V_\mathrm{L} — I_\mathrm{Z}$特性と$I_\mathrm{L} — I_\mathrm{Z}$特性のグラフを作成した。
\end{enumerate}
\begin{figure}[H]
	\centering
	\includegraphics[keepaspectratio, scale=0.5]{const.pdf}
	\caption{ダイオード特性(逆バイアス)}
	\label{fig:const}
\end{figure}

\subsubsection{使用機器}
\begin{table}[htbp]
	\centering
	\caption{使用機器}
	\begin{tabular}{llll}
		\hline
		名称   & 製造元      & 型番                  & 製造番号/管理番号       \\ \hline
		電源装置 & YOKOGAWA & PA1811              & L96-000668      \\
		電圧計  & YOKOGAWA & TYPE 2011 CLASS 0.5 & B-5035.58.1/5   \\
		電流計  & YOKOGAWA & E-11 CLASS 1.0      & B-2042.42.15/15 \\
		電流計  & YOKOGAWA & TYPE 2011 CLASS 0.5 & B-6051.45.6/8   \\
		摺動抵抗 & YOKOGAWA & 不明                  & B-2027.337.4/7  \\ \hline
	\end{tabular}
\end{table}

\subsubsection{結果}
計測した電圧$V_\mathrm{L}$と電流$I_\mathrm{L}$を図\ref{tab:const}に示す。また,この値をグラフに描画したものを図\ref{fig:const1},図\ref{fig:const2}に示す。
\begin{figure}[H]
	\centering
	\includegraphics[keepaspectratio, scale=0.8]{const1.pdf}
	\caption{電流$\,I_\mathrm{L}$ - 電流$\,I_\mathrm{Z}$特性}
	\label{fig:const1}
\end{figure}
\begin{figure}[H]
	\centering
	\includegraphics[keepaspectratio, scale=0.8]{const2.pdf}
	\caption{電流$\,V_\mathrm{L}$ - 電流$\,I_\mathrm{Z}$特性}
	\label{fig:const2}
\end{figure}
\begin{table}[H]
	\centering
	\label{低電圧回路の電圧と電流の関係}
	\begin{tabular}{lll}
		\hline
		$電流I_\mathrm{Z}\,[\mathrm{mA}]$ & $電圧V_\mathrm{L}\,[\mathrm{V}]$ & $電流I_\mathrm{L}[\mathrm{mA}]$ \\ \hline
		2.0                             & 6.98                           & 24.70                         \\
		4.0                             & 7.00                           & 22.30                         \\
		6.0                             & 7.01                           & 20.20                         \\
		8.0                             & 7.04                           & 18.40                         \\
		10.0                            & 7.07                           & 16.20                         \\
		12.0                            & 7.09                           & 14.20                         \\
		14.0                            & 7.11                           & 12.00                         \\
		16.0                            & 7.12                           & 10.10                         \\
		18.0                            & 7.15                           & 8.10                          \\
		20.0                            & 7.18                           & 6.00                          \\
		22.0                            & 7.20                           & 4.00                          \\ \hline
	\end{tabular}
	\label{tab:const}
\end{table}
計測結果より電流$I_\mathrm{Z}$が増加しても電圧$V_\mathrm{L}$はほぼ変化しないのに対し,$I_\mathrm{Z}$が増加すると電流$I_\mathrm{L}$が減少する関係にあることがわかる。

\subsubsection{考察}
\begin{enumerate}[i)]
	\item $R_\mathrm{L}$, $R_\mathrm{Z}$及び合成抵抗$R_\mathrm{A}$の導出
	      $R_\mathrm{L}$, $R_\mathrm{Z}$及び$R_\mathrm{A}$は,次に示す式を用いて導出した。
	      \begin{align}
		      \label{eq:rl}
		      R_\mathrm{L} = \frac{V_\mathrm{L}}{I_\mathrm{L}} \\
		      \label{eq:rz}
		      R_\mathrm{Z} = \frac{V_\mathrm{L}}{I_\mathrm{Z}} \\
		      \label{eq:ra}
		      R_\mathrm{A} = \frac{R_\mathrm{L} \times R_\mathrm{Z}}{R_\mathrm{L} + R_\mathrm{Z}}
	      \end{align}
	      式(\ref{eq:rl}),式(\ref{eq:rz}),式(\ref{eq:ra})より導出した値を表\ref{tab:ra}に示す。
	      可変抵抗$R_\mathrm{L}$を変化させると,合成抵抗$R_\mathrm{A}$が$0.27\,\mathrm{k\Omega}$程度になるよう抵抗$R_\mathrm{Z}$も変化していることがわかる。
	      \begin{table}[H]
		      \centering
		      \caption{$R_\mathrm{L}$, $R_\mathrm{Z}$及び合成抵抗$R_\mathrm{A}$の関係}
		      \begin{tabular}{ccc}
			      \hline
			      抵抗$R_\mathrm{L}\,[\mathrm{k\Omega}]$ & 抵抗$R_\mathrm{Z}\,[\mathrm{k\Omega}]$ & 合成抵抗$R_\mathrm{A}\,[\mathrm{k\Omega}]$ \\ \hline
			      0.28                                 & 3.49                                 & 0.26                                   \\
			      0.31                                 & 1.75                                 & 0.27                                   \\
			      0.35                                 & 1.17                                 & 0.27                                   \\
			      0.38                                 & 0.88                                 & 0.27                                   \\
			      0.44                                 & 0.71                                 & 0.27                                   \\
			      0.50                                 & 0.59                                 & 0.27                                   \\
			      0.59                                 & 0.51                                 & 0.27                                   \\
			      0.70                                 & 0.45                                 & 0.27                                   \\
			      0.88                                 & 0.40                                 & 0.27                                   \\
			      1.20                                 & 0.36                                 & 0.28                                   \\
			      1.80                                 & 0.33                                 & 0.28                                   \\ \hline
		      \end{tabular}
		      \label{tab:ra}
	      \end{table}
	\item $I_\mathrm{Z} - R_\mathrm{L}$特性,$I_\mathrm{Z} - R_\mathrm{Z}$特性の考察
	      抵抗$R$にかかる電圧は,入力電圧からツェナー電圧を引いた$V_\mathrm{i} - V_\mathrm{Z}$となる。よって抵抗$R$に流れる電流=回路全体に流れる電流は一定である。
	      この電流はツェナーダイオード$ZD$と抵抗$R_L$に分流されるが,ツェナーダイオードは流れる電流が変化しても同じツェナー電圧を発生するので,$R_\mathrm{L}$の電流が変化しても出力電圧は不変である。
	      また,図\ref{fig:rlrz}に可変抵抗特性とツェナー抵抗特性を示す。
	      この図より,電流$I_\mathrm{Z}$が増加すると抵抗$R_\mathrm{Z}$は増加し,抵抗$R_\mathrm{L}$が減少する反比例の関係にあることがわかる。

	      \begin{figure}[H]
		      \centering
		      \includegraphics[keepaspectratio, scale=0.8]{ra.pdf}
		      \caption{可変抵抗特性とツェナー抵抗特性}
		      \label{fig:rlrz}
	      \end{figure}
\end{enumerate}





\subsection{太陽電池の発電特性}
\subsubsection{実験方法}
\subsubsection{使用機器}
\subsubsection{結果}
\subsubsection{考察}

\section{結論}

\begin{thebibliography}{9}%参考文献数が10以上の場合は9を99に変更
	%\bibitem{xxx}の引用を本文中で行うには\cite{xxx}と記述。
	\bibitem{d} Semiconductor Components Industries LLC, `` Axial-Lead Glass Passivated Standard Recovery Rectifiers'', \url{https://www.onsemi.jp/pdf/datasheet/1n4001-d.pdf}, 閲覧日 2022年6月27日.
	\bibitem{zd} Semiconductor Components Industried LLC, ``1N4728A - 1N4758A Zener Diodes'', \url{https://rocelec.widen.net/view/pdf/7qfz7dzkx8/ONSM-S-A0003585147-1.pdf?t.download=true&u=5oefqw}, 閲覧日 2022年6月27日.
	\bibitem{rohm} ROHM Co., Ltd., ``ツェナーダイオード(定電圧ダイオード)'', \url{https://www.rohm.co.jp/electronics-basics/diodes/di_what6}, 閲覧日 2022年6月28日.
	\bibitem{revd} 甲南大学, ``半導体/電子デバイス物理'', \url{http://kccn.konan-u.ac.jp/physics/semiconductor/basic/2_6.html}, 閲覧日 2022年6月28日.
\end{thebibliography}


\end{document}

