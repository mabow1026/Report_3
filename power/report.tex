\documentclass[11pt,dvipdfmx]{jarticle}
\usepackage[a4paper,top=3cm,bottom=2cm,left=2cm,right=2cm]{geometry}
\usepackage{mathtools}

\begin{document}
\title{電力と力率 実験前課題}
\author{宮崎 永}
\date{2022/7/13}
\maketitle

\section{原理}
\subsection{瞬時電力}
交流回路の電流と電圧を
\begin{equation}
  i(t) = I_0 \sin (\omega t + \theta_\mathrm{i})
\end{equation}
\begin{equation}
  v(t) = V_0 \sin (\omega t + \theta_\mathrm{v})
\end{equation}
の正弦波によって表される。よってこれらの積
\begin{equation}
  p(t) = i(t)v(t) = I_0 V_0 \sin (\omega t + \theta_\mathrm{i}) \sin (\omega t + \theta_\mathrm{v})
\end{equation}
を瞬時電力という。
\subsection{有効電力}
有効電力は,負荷で実際に消費される電力のことであり,これを次の式で表す。
\begin{equation}
  P = V I \cos \theta [\mathrm{W}]
\end{equation}
\subsection{皮相電力}
皮相電力は,電源から送られた全体の電力のことであり,これを次の式で表す。
\begin{equation}
  S = V I [\mathrm{VA}]
\end{equation}
\subsection{無効電力}
無効電力は,負荷で消費されない電力のことであり,これを次の式で表す。
\begin{equation}
  Q = V I \sin \theta [\mathrm{var}]
\end{equation}
または
\begin{equation}
  Q = V I \sqrt{1 - \cos^2 \theta} [\mathrm{var}]
\end{equation}
\subsection{力率}
力率$\cos \theta$は,電源から送られた電力が負荷で消費される割合のことであり,
\begin{equation}
  \cos \theta = \frac{有効電力}{皮相電力} = \frac{P}{VI}
\end{equation}
で表される。

\section{数式展開}
\begin{align}
  p = ei = -|Z|I^2 \cos(2 \omega t + \angle Z) + |Z|I^2 \cos\angle Z
\end{align}
\begin{align}
  p_\mathrm{a} = |Z|I^2 \cos \angle Z(1 - \cos 2 \omega t)
\end{align}
\begin{align}
  p_\mathrm{r} = ei = |Z|I-2 \sin \angle Z \sin 2 \omega t
\end{align}
\begin{align}
  e = \sqrt{2} E \sin (\omega t + \angle Z)
\end{align}



\end{document}