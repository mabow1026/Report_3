\documentclass[11pt,dvipdfmx]{jarticle}

\usepackage{eee}
\usepackage{subfig}
\usepackage{here}

\renewcommand{\labelenumi}{\alph{enumi}}
\renewcommand{\labelenumii}{\roman{enumii}}

\begin{document}
% トップページを書く
\begin{jikkenTitle}
	\gakunen{3} % 学年を記述。この行で全体の枠を表示
	\numTitle{3}{電力と力率} % 実験番号、タイトルを記述
	\subTitle{} % サブタイトルがあれば記述
	\jikkenbi{令和4年7月14日(木)} % 実験日を記述
	\jikkenbiII{令和4年7月21日(木)} % 実験日を記述(二日目がある場合。ない場合はこの行をコメントアウト)
	\kyoudou{3325 谷下 文紀 3329 野島 奏一朗 3337吉野 曹生} % 共同実験者名を記述
	\kyoudouII{} % その他の共同実験者名を記述
	\yoteibi{/  }% 予定日を記述
	\yoteibiII{}% 予定日2を記述
	\yoteibiIII{}% 予定日3を記述
	\hanNumberName{4}{3333}{宮崎 永} % 班番号・学生番号・氏名を記述。この行でタイトルページの描画を終了
\end{jikkenTitle}

\section{目的}
本実験では
\begin{itemize}
	\item 単相交流回路における電圧・電流・電力・力率を測定するための結線方法を学ぶ。
	\item 単相電力計と力率計の扱い方を習得する。
	\item 有効電力と力率、皮相電力と無効電力に関する理解を深める
\end{itemize}
ことを目的とする。

\section{原理}
\subsection{瞬時電力}
インピーダンス$\dot{Z}\,[\Omega]$へ印加された時刻$t\,[\mathrm{s}]$における交流電圧$v(t)\,[\mathrm{V}]$と、$\dot{Z}$に流れる交流電流$i(t)\,[\mathrm{A}]$がそれぞれ次式で表されるとする。
\begin{eqnarray}
	v(t) &=& V_m\sin(\omega t+\theta_V)\\
	i(t) &=& I_m\sin(\omega t + \theta_I)
\end{eqnarray}
ここで、$V_m$、$I_m$は最大値、$\omega\,[\mathrm{rad/s}]$は角周波数、$\theta_V\,[\mathrm{rad}]$と$\theta_I\,[\mathrm{rad}]$はそれぞれの位相である。
この$v(t)$と$i(t)$の積を瞬時電力$p(t)$と呼び、次式で表される。
\begin{eqnarray}
	p(t)&=& v(t)i(t)\nonumber\\
	&=&V_mI_m\sin(\omega t+\theta_V)\sin(\omega t + \theta_I)\nonumber\\
	&=& \frac{V_mI_m}{2}\Bigl(\cos(2\omega t + \theta_I+\theta_V)+\cos(\theta_I - \theta_V)\Bigr)
	\label{eq:ip}
\end{eqnarray}


\subsection{有効電力と力率}
\weq{ip}は$v(t)$や$i(t)$の2倍の角速度を持つ周期関数であることが確認できる。
そのため、時間的な平均を算出することができ、この値を有効電力$P\,[\mathrm{W}]$と呼ぶ。
\begin{eqnarray}
	P &=& \frac{1}{T}\int_0^T \frac{V_mI_m}{2}\Bigl(\cos(2\omega t + \theta_I+\theta_V)+\cos(\theta_I - \theta_V)\Bigr) dt\nonumber\\
	&=&\frac{V_mI_m}{2}\cos(\theta_I - \theta_V)
\end{eqnarray}

この上式が得られたとき、交流回路における実効値表現に置き換えると
\begin{equation}
	P = VI\cos\theta
	\label{eq:power}
\end{equation}
を得ることができる。ここで、$V$、$I$はそれぞれの実効値、$\theta=\theta_I-\theta_V$である。
\weq{power}の右辺は電圧と電流の実効値の積と、$\cos\theta$から構成されている。$\theta$は$\dot{Z}$の実部(抵抗)と虚部(リアクタンス)の比によって決定される値であり、
\begin{equation}
	-\frac{\pi}{2}\leq\theta\leq\frac{\pi}{2}
\end{equation}
であるので、
\begin{equation}
	0\leq\cos\theta\leq 1
\end{equation}
の不等式が成立する。

以上の関係から、インピーダンス$\dot{Z}$の端子電圧と流れる電流値の積とは必ずしも等しくなく、有効に消費される電力の比が$\cos\theta$に相当することが分かる。
この比として見なせる$\cos\theta$を力率、$\theta$を力率角と呼ぶ。

\subsection{無効電力と皮相電力}
\weq{ip}において、インピーダンスがリアクタンス成分のみ($\dot{Z}=jX$)の場合について考える。
この時、電圧と電流の位相差$\theta_I-\theta_V$は$\pm\pi/2$となり、括弧内の第二項の値は0となる。
従って、瞬時電力$p(t)$の振る舞いは平均値が0の正弦波(あるいは余弦波)になることが分かる。
これは、電源から負荷へ、負荷から電源へ電力供給が交互に行われていることを示し、電力として消費されず仕事をしない。
この電力を無効電力$Q$とよび、単位には$\mathrm{var}$(バール)を用い、次式で計算される。
\begin{equation}
	Q = VI \sin \theta
\end{equation}

電圧の実効値と電流の実効値の積$VI$は、インピーダンス$\dot{Z}$が純抵抗(リアクタンス$X=0$)の場合にのみ有効電力と等しくなり、それ以外の場合では$VI> P$となる。
この、見かけ上の電力を皮相電力$S$とよび、単位には$\mathrm{VA}$(ボルトアンペア)を用いる。
また、皮相電力と有効電力、無効電力には次の関係が成り立つ。
\begin{eqnarray}
	S &=& VI\nonumber\\
	&=& \sqrt{P^2+Q^2}
\end{eqnarray}
\section{事前学習}
\subsection{瞬時電力}
交流回路の電流と電圧を
\begin{equation}
	i(t) = I_0 \sin (\omega t + \theta_\mathrm{i})
\end{equation}
\begin{equation}
	v(t) = V_0 \sin (\omega t + \theta_\mathrm{v})
\end{equation}
の正弦波によって表される。よってこれらの積
\begin{equation}
	p(t) = i(t)v(t) = I_0 V_0 \sin (\omega t + \theta_\mathrm{i}) \sin (\omega t + \theta_\mathrm{v})
\end{equation}
を瞬時電力という。
\subsection{有効電力}
有効電力は,負荷で実際に消費される電力のことであり,これを次の式で表す。
\begin{equation}
	P = V I \cos \theta [\mathrm{W}]
\end{equation}
\subsection{皮相電力}
皮相電力は,電源から送られた全体の電力のことであり,これを次の式で表す。
\begin{equation}
	S = V I [\mathrm{VA}]
\end{equation}
\subsection{無効電力}
無効電力は,負荷で消費されない電力のことであり,これを次の式で表す。
\begin{equation}
	Q = V I \sin \theta [\mathrm{var}]
\end{equation}
または
\begin{equation}
	Q = V I \sqrt{1 - \cos^2 \theta} [\mathrm{var}]
\end{equation}
\subsection{力率}
力率$\cos \theta$は,電源から送られた電力が負荷で消費される割合のことであり,
\begin{equation}
	\cos \theta = \frac{有効電力}{皮相電力} = \frac{P}{VI}
\end{equation}
で表される。

\section{数式展開}
\begin{align}
	p(t) \quad & = \quad	e(t) \times i(t) \nonumber                                   \\
	           & = \quad 2 I^{2}|Z| \sin (\omega t+\angle Z) \sin \omega t \nonumber  \\
	           & = \quad -|Z| I^{2} \cos (2 \omega t+\angle Z)+I^{2}|Z| \cos \angle Z
\end{align}
\begin{align}
	P_\mathrm{r}(t) \quad & = \quad e_{r}(t) \times i(t) \nonumber                                               \\
	                      & = \quad 2 R I^{2}|Z| \sin \omega t \nonumber                                         \\
	                      & = \quad 2|Z| I^{2} \cos (\angle Z) \times\left(1-\cos ^{2} \omega t\right) \nonumber \\
	                      & = \quad 2 I^{2}|Z| \cos (\angle Z)\left(1-\frac{1+\cos 2 \theta}{2}\right) \nonumber \\
	                      & = \quad I^{2}|Z| \cos (\angle Z)(1-\cos \omega t)
\end{align}
\begin{align}
	P_\mathrm{r}(t) \quad & = \quad e_{r}(t) \times i(t) \nonumber                                                                                                \\
	                      & = \quad \left(\sqrt{2} I \sin \left(\omega t \pm \frac{\pi}{2}\right) X\right) \times \sqrt{2} I \sin \omega t \nonumber              \\
	                      & = \quad 2 X I^{2} \sin \left(\omega t \pm \frac{\pi}{2}\right) \sin \omega t \nonumber                                                \\
	                      & = \quad 2 X I^{2}\left(-\frac{1}{2} \cos \left(\omega t \pm \frac{\pi}{2}\right)-\cos \left(\pm \frac{\pi}{2}\right)\right. \nonumber \\
	                      & = \quad I^{2} X \cos \left(\sin t \pm \frac{\pi}{2}\right) \nonumber                                                                  \\
	                      & = \quad -|Z| I^{2} \sin \angle Z \times \cos \left(2 \omega t \mp \frac{\pi}{2}\right)
\end{align}
\begin{align}
	P_{r} \quad & = \quad |Z| I^{2}=E I \nonumber                          \\
	e           & = \quad \sqrt{2}|Z| I \sin (\omega t+\angle Z) \nonumber \\
	e           & = \quad \sqrt{2} E \sin (\omega t+\angle Z)
\end{align}
\newpage

\section{方法}
\subsection{使用器具}
今回の実験で使用した器具を表\ref{tab:kiki}に示す
\begin{table}[H]
	\caption{使用器具(機器)}
	\begin{tabular}{lllll}
		\hline
		名称     & 製造元      & 型番               & 製造番号/管理番号       & 備考(レンジ等)                                     \\ \hline
		電圧計    & YEW      & M9G1723          & B-3039.44.10/10 & 150/300$\,[\mathrm{V}]$                      \\
		電流計    & YEW      & M9G846           & -               & 2/5/10/20$\,[\mathrm{A}]$                    \\
		電力計    & YOKOGAWA & YEW2041 120/240V & B-3041.H1.1/2   & 5/25$\,[\mathrm{A}]$ 120/240$\,[\mathrm{V}]$ \\
		力率計    & YOKOGAWA & TYPE 2039        & -               & 5/25$\,[\mathrm{A}]$                         \\
		綜合負荷装置 & 山菱電機株式会社 & UI-0090          & B-3067.46.2/2   & 力率設定 0.1 - 0.98                              \\
		スライダック & 東芝       & SLIDAC SD120     & B-5031.H1.2/3   & 出力電圧 0 - 130$\,[\mathrm{V}]$                 \\ \hline
	\end{tabular}
	\label{tab:kiki}
\end{table}

\subsection{実験手順}
\begin{figure}[H]
	\centering
	\includegraphics[keepaspectratio, scale=0.65]{power.pdf}
	\caption{実験回路図}
	\label{fig:power}
\end{figure}

\section{結果}
\section{考察}
\section{結論}

\begin{thebibliography}{9}%参考文献数が10以上の場合は9を99に変更
	%\bibitem{xxx}の引用を本文中で行うには\cite{xxx}と記述。
	\bibitem{a} 著者名, 書名,出版社,発行年.
\end{thebibliography}


\end{document}